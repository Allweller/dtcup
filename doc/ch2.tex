\chapter{第十届大唐杯A组省赛第二场}
\fancyhead[L]{\color{H6}\kaishu\faIcon{atom}\;第十届「大唐杯」省赛A组真题第二套}
\fancyhead[R]{\color{H6}\kaishu\rightmark\,}

\date{2023年4月15日}{}{\href{https://qm.qq.com/q/UPbGudx8cK}{\textbf{才疏学浅的小熊}}}
{}
{https://ctan.org/pkg/litesolution}{CTAN}
{https://github.com/xubenshan/dtcup}{GitHub}

源码已上传至Github,有需要的同学可以自取,见右侧二维码。

\section{单选题(共30道题,共90分)}



\begin{choice}{D}[]
	C-V2X中,车与RSU之间的通信接口是
	\begin{tasks}(4)
		\task Ng
		\task Xn
		\task Uu
		\task PC5
	\end{tasks}
\end{choice}


\begin{choice}{A}[]
	在5G NR网络中终端在初始接入过程中,基站向核心网发的第一条信息是
	\begin{tasks}(2)
		\task Initial UE Message
		\task Uplink NAS Transfer
		\task UE Capability Information
		\task RRC Setup Complete
	\end{tasks}
\end{choice}




\begin{choice}{D}[]
	NR系统中,全双工的最大难点在于
	\begin{tasks}(2)
		\task 滤波
		\task 解调
		\task 调制
		\task 干扰抑制
	\end{tasks}
\end{choice}


\begin{choice}{B}[]
	IPMT产品生命周期管理阶段的周期长短取决于什么因素
	\begin{tasks}(2)
		\task 产品亏损数额
		\task 产品的投资回报率
		\task 产品盈利数额
		\task 产品用户数
	\end{tasks}
\end{choice}

% \begin{solution}
%     R15版本是第一代5G的版本。PDSCH(物理下行数据信道)支持的最大调制方式为256QAM。在R16版本中又增加了1024QAM调制。

%     补充:下行不同信道的调制方式:
%     \begin{itemize}
%         \item PBCH(物理广播信道)QPSK
%         \item PDCCH(物理下行控制信道)QPSK
%         \item PDSCH(物理下行数据信道)QPSK,16QAM,64QAM,256QAM,1024QAM

%               \sokka{D}
%     \end{itemize}
% \end{solution}
\begin{choice}{D}[]
	对于每个服务小区,基站可以通过专用RRC信令给UE配置多个DL BWP和多个UL BWP,最多各配多少个
	\begin{tasks}(1)
		\task 12
		\task 2
		\task 8
		\task 4
	\end{tasks}
\end{choice}

\begin{choice}{B}[]
	5G SA场景下,Uu口控制面协议从上到下的次序依次是
	\begin{tasks}(2)
		\task SDAP-PDCP-RLC-MAC-PHY
		\task RRC-PDCP-RLC-MAC-PHY
		\task IP-PDCP-RLC-MAC-PHY
		\task SCTP-PDCP-RLC-MAC-PHY
	\end{tasks}
\end{choice}

\begin{solution}
	用户面独有SDAP。
\end{solution}


\begin{choice}{A}[]
	在通信系统中,正常情况下天馈系统驻波比的正常范围是
	\begin{tasks}(2)
		\task 1-1.5
		\task 0-1.5
		\task 1-2
		\task 2-3
	\end{tasks}
\end{choice}


\begin{choice}{A}[]
	在5G网络架构中,以下选项哪一项是AMF的功能
	\begin{tasks}(2)
		\task 注册管理
		\task 无线资源分配
		\task 会话的建立修改删除
		\task 下行数据的通知
	\end{tasks}
\end{choice}
%    \switchcolumn\centering
%    \vfill
%    \begin{tikzpicture}[decoration={markings,mark=between positions .2 and .8 step 18mm with {\arrow{stealth}}}]
%        \draw [very thick] (0,1.5)--(0,-1.5);
%        \draw [thick,postaction=decorate] (-1,0)--(0,0.5)--(2,0.8);
%        \draw [thick] (0,0.5)--(0.8,0);
%        \draw [thick,postaction=decorate] (-1,0)--(0,-0.5)--(2,0.8);
%        \draw [very thick] (0,0)--(1.8,0);
%        \draw [very thick] (2,1.8)--(2,-1.8);
%        \fill [pattern=north west lines] (0,0) rectangle (1.8,-0.2);
%        \fill [pattern=north east lines] (2,1.8) rectangle (2.2,-1.8);
%        \node [anchor=east] at (-1,0) {$S$} node [anchor=south east] at (0,0.5) {$S_1$} node [anchor=north east] at (0,-0.5) {$S_2$};
%        \node [anchor=north] at (0.8,-0.2) {$M$} node [anchor=west] at (2.2,0.8) {$P$} node [anchor=north] at (2.1,-1.8) {$E$};
%        \node at (-1,0) {$\times$};
%    \end{tikzpicture}
%    \vfill
%\end{paracol}
%\vspace{-.75em}


\begin{choice}{A}[]
	关于基站节能,以下说法正确的是
	\begin{tasks}(1)
		\task 基站应该在保证用户体验的前提下,尽可能多的节能。
		\task 由于节能必然影响用户体验,因此该功能禁用。
		\task 节能的优先级高于用户体验
		\task 节能不需要考虑任何业务影响
	\end{tasks}
\end{choice}


\begin{choice}{D}[]
	5G NR初始随机接入过程中哪条消息表示竞争解决
	\begin{tasks}(4)
		\task MSG3
		\task MSG2
		\task MSG5
		\task MSG4
	\end{tasks}
\end{choice}

\begin{choice}{D}[]
	下列传输介质中,传输效率最高的是
	\begin{tasks}(4)
		\task 无线介质
		\task 同轴电缆
		\task 网线
		\task 光纤
	\end{tasks}
\end{choice}


\begin{choice}{D}[]
	700M组网采用的子帧配比是
	\begin{tasks}(1)
		\task DDDSUDDSUU S时隙6:4:4
		\task DDDSUDDSUU S时隙10:2:2
		\task DDDDDDDSUU S时隙6:4:4
		\task 下行DDDDDDDDDD上行UUUUUUUUUU
	\end{tasks}
\end{choice}


\begin{choice}{B}[]
	5G NR系统中,一个SS/PBCH block包含的OFDM symbols个数是
	\begin{tasks}(4)
		\task 2
		\task 4
		\task 3
		\task 1
	\end{tasks}
\end{choice}


\begin{choice}{D}[]
	以下选项中,属于GBDT和AdaBoost区别的是
	\begin{tasks}(1)
		\task 二者无区别
		\task 预测器不同
		\task GBDT在每个迭代中调整实例权重,而AdaBoost让新预测器针对前一个预测器的残差进行拟合。
		\task AdaBoost在每个迭代中调整实例权重,而GBDT让新的预测器针对前一个预测器的残差进行拟合。
	\end{tasks}
\end{choice}

\begin{choice}{A}[]
	根据电话号码编号计划规则,其中中国的国家号码标识为
	\begin{tasks}(4)
		\task 86
		\task 10
		\task 88
		\task 00
	\end{tasks}
\end{choice}

\begin{choice}{C}[]
	在5G NR帧结构中,当SCS子载波间隔为30kHZ时一个子帧包含的符号数为
	\begin{tasks}(2)
		\task 48个
		\task 56个
		\task 28个
		\task 14个
	\end{tasks}
\end{choice}
\begin{solution}
	SCS子载波间隔为30Khz时,一个子帧包含2个时隙,1个时隙包含14个OFDM符号,则1个子帧包含28个符号数。
\end{solution}


\begin{choice}{B}[]
	以下采用的基于竞争的随机接入的是
	\begin{tasks}(2)
		\task RRC重建
		\task 波束失败恢复
		\task 上行失步态UE下行数据到达
		\task 切换
	\end{tasks}
\end{choice}

\begin{choice}{D}[]
	5G NR系统中,UE可通过哪条系统消息获取小区重选公共信息
	\begin{tasks}(4)
		\task SIB4
		\task SIB1
		\task SIB3
		\task SIB2
	\end{tasks}
\end{choice}


\begin{choice}{A}[]
	5G NR协议栈,负责QoS流到DRB的映射的协议层为
	\begin{tasks}(1)
		\task SDAP
		\task RRC
		\task PLC
		\task PDCP
	\end{tasks}
\end{choice}



\begin{choice}{B}[]
	C-V2X中,车辆与基站之间直连通信接口为()
	\begin{tasks}(2)
		\task Xn
		\task Uu
		\task PC5
		\task Ng
	\end{tasks}
\end{choice}



\begin{choice}{C}[]
	5G网络基本架构,AMF与gNB之间的接口是
	\begin{tasks}(4)
		\task Xn
		\task NG-U
		\task NG-C
		\task N11
	\end{tasks}
\end{choice}




\begin{choice}{A}[]
	5G NR中当子载波间隔为60KHz时,一个时隙的时间长度是多少
	\begin{tasks}(4)
		\task 0.25ms
		\task 1ms
		\task 0.5ms
		\task 0.125ms
	\end{tasks}
\end{choice}

\begin{choice}{C}[]
	以下选项中,不属于AI赋能5G方面应用的是
	\begin{tasks}(4)
		\task 网络优化
		\task 流量预测
		\task 机器翻译
		\task 异常检测
	\end{tasks}
\end{choice}

\begin{choice}{C}[]
	以下选项中,哪一个选项属于财务会计核算成本的方法
	\begin{tasks}(2)
		\task 作业变动法
		\task 变动成本法
		\task 完全成本法
		\task 作业变动成本法
	\end{tasks}
\end{choice}


\begin{choice}{C}[]
	将网络设备分离为单独的控制设备及转发设备的网络技术是()
	\begin{tasks}(1)
		\task NFV
		\task 切片
		\task SDN
		\task MEC
	\end{tasks}
\end{choice}


\begin{choice}{B}[]
	在5G NR PRACH规划中,Format0-3对应ZC序列长度为
	\begin{tasks}(2)
		\task 138
		\task 839
		\task 838
		\task 139
	\end{tasks}
\end{choice}



\begin{choice}{B}[]
	NR 3.5G组网时,终端目前实现的最大流数为
	\begin{tasks}(2)
		\task UL 1流 DL 2流
		\task UL 2流 DL 2流
		\task UL 2流 DL 1流
		\task UL 1流 DL 1流
	\end{tasks}
\end{choice}


\begin{choice}{D}[]
	以下关于产品全成本概念下的变动成本表示正确的是
	\begin{tasks}(1)
		\task 产品变动成本=产品生产变动成本+产品服务变动成本
		\task 产品变动成本=产品销售变动成本+产品服务变动成本
		\task 产品变动成本=产品生产变动成本+产品销售变动成本
		\task 产品变动成本=产品生产变动成本+产品销售变动成本+产品服务变动成本
	\end{tasks}
\end{choice}


\begin{choice}{C}[]
	以下关于技术重用的说法,错误的是
	\begin{tasks}(1)
		\task 通过技术重用可优化产品固定成本
		\task  技术重用可提高产品的可靠性
		\task 工程领域的技术重用与高校系统的查重概念是一致的
		\task 技术重用可缩短产品的开发周期
	\end{tasks}
\end{choice}


\begin{choice}{C}[]
	以下选项中关于复杂技术问题和复杂工程问题描述错误的是
	\begin{tasks}(1)
		\task 解决复杂技术问题以技术性的突破为目的
		\task 解决复杂工程问题将技术性突破作为手段
		\task 解决复杂技术问题以获得高价值经济回报作为最终目的
		\task 解决复杂工程问题以获得价值经济回报为最终目的
	\end{tasks}
\end{choice}

\section{多选题(共20小题,共80分)}

\begin{choice}{\;ABCD\;}[]
	属于5G超密集组网应用场景有
	\begin{tasks}(2)
		\task 商业中心
		\task 大型活动场馆
		\task 密集住宅区
		\task 交通枢纽
	\end{tasks}
\end{choice}

\begin{choice}{\;ABCD\;}[]
	5G NR系统中,以下哪些原因会导致RRC连接重建
	\begin{tasks}(2)
		\task 重配置失败
		\task 切换失败
		\task 完保校验失败
		\task 检测到RLF
	\end{tasks}
\end{choice}

\begin{choice}{\;ACD\;}[]
	SIB1中包含的关键信息有
	\begin{tasks}(2)
		\task 小区是否禁止接入
		\task SFN高6位
		\task 小区选择信息
		\task 小区服务的PLMN列表
	\end{tasks}
\end{choice}
\begin{solution}
  B选项属于MIB的关键信息。
\end{solution}
\begin{choice}{\;ABC\;}[]
	以下哪些选项属于NR-V2X可支持传输模式
	\begin{tasks}(1)
		\task 组播
		\task 广播
		\task 单播
		\task 多播
	\end{tasks}
\end{choice}


\begin{choice}{\;ACD\;}[]
	5G网络中物理下行参考信号DM-RS可能伴随哪些物理信道传输
	\begin{tasks}(4)
		\task PDCCH
		\task PUCCH
		\task PBCH
		\task PDSCH
	\end{tasks}
\end{choice}

\begin{choice}{\;ACD\;}[]
	关于MEC部署的优点,以下说法正确的是
	\begin{tasks}(2)
		\task 大连接;数据网关
		\task 各业务逻辑隔离
		\task 热点区域;内容增加
		\task 低时延;控制和数据业务下沉
	\end{tasks}
\end{choice}

\begin{choice}{\;ABD\;}[]
	下面有关5G室分解决方案中说法正确的是
	\begin{tasks}(1)
		\task 针对高容量,大面积。高价值的场景可采用数字化室分(皮站)进行解决。
		\task 针对话务高发但人员密集度相对较低的区域可采用一体化小基站(飞站)进行解决。
		\task 传统室内分布解决方案一定都是无源的
		\task Slsite解决方案适用于中等容量,多隔断环境复杂,中等价值场景
	\end{tasks}
\end{choice}


\begin{choice}{\;BD\;}[]
	基站的网关地址为120.1.1.25,掩码为255.255.255.248,则以下地址可以作为基站IP地址的选项为
	\begin{tasks}(2)
		\task 120.1.1.25
		\task 120.1.1.27
		\task 120.1.1.24
		\task 120.1.1.29
	\end{tasks}
\end{choice}
\begin{solution}
 网关地址和子网掩码相与得到120.1.1.24。广播地址为120.1.1.31。可用IP地址为24到31之间,记得要除去网关地址。
\end{solution}

\begin{choice}{\;BD\;}[]
	5G系统消息中,哪些系统消息可以是周期广播也可以是订阅广播
	\begin{tasks}(2)
		\task MIB
		\task SIB3
		\task SIB1
		\task SIB2
	\end{tasks}
\end{choice}


\begin{choice}{\;ACD\;}[]
	以下选项中,属于Boosting提升法的选项是
	\begin{tasks}(1)
		\task 随机森林
		\task cart树
		\task AdaBoost
		\task Gradient Boosting
	\end{tasks}
\end{choice}


\begin{choice}{\;ABCD\;}[]
	根据产品研发流程各环节的成本分析,属于产品固定成本的为
	\begin{tasks}(2)
		\task 产品测试验证环节的成本
		\task 技术预研环节的成本
		\task 产品策划环节的成本
		\task 产品开发环节的成本
	\end{tasks}
\end{choice}

\begin{choice}{\;ABCD\;}[]
	以下频段中,5G网络建设采用子载波间隔15kHz组网方案的是
	\begin{tasks}(1)
		\task3.5G
		\task 2.6G
		\task 4.9G
		\task 700M
	\end{tasks}
\end{choice}

\begin{choice}{\;CD\;}[]
	下面有关5G频谱共享,说法正确的是
	\begin{tasks}(1)
		\task 只共享基站不共享传输核心网
		\task NSA共享,性能领先SA
		\task 有独立载波和共享载波两种共享方式
		\task 分SA架构和NSA架构
	\end{tasks}
\end{choice}

\begin{choice}{\;ABC\;}[]
	下面哪个环节产生的成本属于变动成本
	\begin{tasks}(2)
		\task 生产安装环节
		\task 装配生产环节
		\task 采购管理环节
		\task 研发流程环节
	\end{tasks}
\end{choice}

\begin{choice}{\;ABD\;}[]
	有关C-V2X说法正确的是
	\begin{tasks}(2)
		\task X可以代表人车网等
		\task 可以基于LTE技术实现
		\task  基于DSRC的技术实现
		\task 可以基于NR技术实现
	\end{tasks}
\end{choice}

\begin{choice}{\;ABD\;}[]
	5G系统中,当子载波间隔为1.25KHz和5KHz时,可以配置的Format格式有
	\begin{tasks}(4)
		\task 0
		\task 1
		\task 4
		\task 2
	\end{tasks}
\end{choice}
\begin{solution}
format格式有{0,1,2,3}
\end{solution}
\begin{choice}{\;ABC\;}[]
	以下选项中,属于决策树求解算法的是
	\begin{tasks}(4)
		\task  C4.5
		\task ID3
		\task CART算法
		\task 梯度下降法
	\end{tasks}
\end{choice}

\begin{choice}{\;ACD\;}[]
	5ms帧结构配比DDDSUDDSUU,S配比为10:2:2,下面说法正确的是
	\begin{tasks}(2)
		\task 2.5ms双周期
		\task 5ms单周期
		\task 是FDD双工
		\task 特殊时隙GP的符号为2
	\end{tasks}
\end{choice}

\begin{choice}{\;ABD\;}[]
	下面有关毫米波通信的说法,正确的是
	\begin{tasks}(2)
		\task 波束集中,提高能效
		\task 方向性好,受干扰影响小
		\task 30-300GHz
		\task 受空气和雨水等影响较大
	\end{tasks}
\end{choice}

\begin{choice}{\;ABCD\;}[]
	以下选项中,会影响用户业务速率的是
	\begin{tasks}(1)
		\task 流数
		\task 调制方式
		\task 分配的带宽
		\task 误块率
	\end{tasks}
\end{choice}




\section{判断题(共10小题,共30分)}

\begin{choice}{\;错误\;}[]
	SCS越大,一个子帧时隙数越多,每个符号时间越长
\end{choice}


\begin{choice}{\;正确\;}[]
	控制与转发分离:控制面进行统一的策略控制,转发面更关注转发功能的实现,以软件来驱动,实现了软硬件的解耦
\end{choice}

\begin{choice}{\;正确\;}[]
	Boosting方法是将几个弱分类器结合成一个强学习器的集成方法
\end{choice}

\begin{choice}{\;正确\;}[]
	新产品的需求由外部需求和内部需求两大类所组成
\end{choice}

\begin{choice}{\;错误\;}[]
	SSB可配置的波束数量和子帧配比有关系
\end{choice}

\begin{choice}{\;错误\;}[]
	可以用SSB软调完全代替硬调方位角和下倾角

\end{choice}


\begin{choice}{\;正确\;}[]
	UE测量报告已经触发,但没收到基站下发切换命令,原因可能是目标基站拥塞

\end{choice}

\begin{choice}{\;错误\;}[]
	5G NR帧结构Normal CP一个子帧固定包含2个时隙

\end{choice}

\begin{choice}{\;错误\;}[]

	5G NR中信令承载DRB被定义为用于传递RRC和NAS消息的无线承载
\end{choice}

\begin{solution}
	5G NR中信令承载SRB被定义为用于传递RRC和NAS消息的无线承载
\end{solution}

\begin{choice}{\;正确\;}[]
	CSI参考信号(CSI-RS)用于信道测量和反馈,而解调参考信号(DMRS)用于数据解调

\end{choice}











