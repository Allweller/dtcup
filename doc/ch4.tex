\chapter{第十届大唐杯B组省赛第二场}
\fancyhead[L]{\color{H6}\kaishu\faIcon{atom}\;第十届「大唐杯」省赛B组真题第二套}
\fancyhead[R]{\color{H6}\kaishu\rightmark\,}

\date{2023年4月16日}{}{\href{https://qm.qq.com/q/UPbGudx8cK}{\textbf{才疏学浅的小熊}}}
{}
{https://ctan.org/pkg/litesolution}{CTAN}
{https://github.com/xubenshan}{GitHub}

注:答案仅供参考,若有问题,请联系我。QQ号:1782622988

\section{单选题(共30道题,共90分)}



\begin{choice}{B}[节能]
    关于基站节能功能,以下说法正确的是
    \begin{tasks}(1)
        \task 载波关断节能不会影响用户接入功能。
        \task 通道关断节能应该在业务量较多的时间段开启。
        \task 符号关断节能可以全天开启。
        \task 以上三种说法均错误。
    \end{tasks}
\end{choice}


\begin{choice}{B}[5G无线技术]
    中国移动目前5G配置5ms单周期,时隙配比7:2:1(特殊6:4:4),此时上行满调度(加特殊子帧)是
    \begin{tasks}(4)
        \task 400
        \task 600
        \task 1000
        \task 800
    \end{tasks}
\end{choice}




\begin{choice}{A}[5G无线技术]
    5G NR中,若SCS为30kHz,则一个RB的宽度是(\qquad)kHz
    \begin{tasks}(4)
        \task 180
        \task 360
        \task 720
        \task 1440
    \end{tasks}
\end{choice}


\begin{choice}{D}[5G协议与信令]
    以下选项中,不属于决策树求解算法的选项是
    \begin{tasks}(4)
        \task CART算法
        \task C4.5
        \task ID3
        \task 梯度下降法
    \end{tasks}
\end{choice}

% \begin{solution}
%     R15版本是第一代5G的版本。PDSCH(物理下行数据信道)支持的最大调制方式为256QAM。在R16版本中又增加了1024QAM调制。

%     补充:下行不同信道的调制方式:
%     \begin{itemize}
%         \item PBCH(物理广播信道)QPSK
%         \item PDCCH(物理下行控制信道)QPSK
%         \item PDSCH(物理下行数据信道)QPSK,16QAM,64QAM,256QAM,1024QAM

%               \sokka{D}
%     \end{itemize}
% \end{solution}
\begin{choice}{B}[节能]
    NR 700M组网时,SSB波束配置最大数为
    \begin{tasks}(4)
        \task 2
        \task 8
        \task 7
        \task 4
    \end{tasks}
\end{choice}

\begin{choice}{D}[]
    NR系统中,全双工的最大难点在于
    \begin{tasks}(4)
        \task 滤波
        \task 干扰抑制
        \task 解调
        \task 调制
    \end{tasks}
\end{choice}




\begin{choice}{D}[最大分辨力]
    大唐5G设备天线安装时,方位角以及下倾角要求偏差度为
    \begin{tasks}(2)
        \task 正负5度,正负5度
        \task 正负5度,正负1度
        \task 正负1度,正负1度
        \task 正负1度,正负5度
    \end{tasks}
\end{choice}


\begin{choice}{B}[双缝干涉]
    产品开发测试中分为不同的环节,其中黑箱测试的主要目的是
    \begin{tasks}(1)
        \task 验证产品内部的实现方式
        \task 实现透明测试
        \task 检测客户需求是否实现
        \task 验证设计规范的实现
    \end{tasks}
\end{choice}
%    \switchcolumn\centering
%    \vfill
%    \begin{tikzpicture}[decoration={markings,mark=between positions .2 and .8 step 18mm with {\arrow{stealth}}}]
%        \draw [very thick] (0,1.5)--(0,-1.5);
%        \draw [thick,postaction=decorate] (-1,0)--(0,0.5)--(2,0.8);
%        \draw [thick] (0,0.5)--(0.8,0);
%        \draw [thick,postaction=decorate] (-1,0)--(0,-0.5)--(2,0.8);
%        \draw [very thick] (0,0)--(1.8,0);
%        \draw [very thick] (2,1.8)--(2,-1.8);
%        \fill [pattern=north west lines] (0,0) rectangle (1.8,-0.2);
%        \fill [pattern=north east lines] (2,1.8) rectangle (2.2,-1.8);
%        \node [anchor=east] at (-1,0) {$S$} node [anchor=south east] at (0,0.5) {$S_1$} node [anchor=north east] at (0,-0.5) {$S_2$};
%        \node [anchor=north] at (0.8,-0.2) {$M$} node [anchor=west] at (2.2,0.8) {$P$} node [anchor=north] at (2.1,-1.8) {$E$};
%        \node at (-1,0) {$\times$};
%    \end{tikzpicture}
%    \vfill
%\end{paracol}
%\vspace{-.75em}


\begin{choice}{A}[迈克尔逊干涉仪]
    在5G NR中PBCH信道相邻DM-RS在频域上的间隔的子载波数是
    \begin{tasks}(4)
        \task 2
        \task 4
        \task 8
        \task 6
    \end{tasks}
\end{choice}


\begin{choice}{A}[光栅]
    以下选项中,属于正常的产品销售定价策略的为
    \begin{tasks}(1)
        \task 产品销售单价低于单位产品变动成本与单位产品对应分摊的固定成本之和
        \task 产品销售单价高于单位产品变动成本,低于单位变动成本与单位产品对应分摊的固定成本之和
        \task 产品销售单价等于单位产品变动成本
        \task 产品销售单价高于单位产品中变动成本与单位产品对应分摊的固定成本之和
    \end{tasks}
\end{choice}

\begin{choice}{A}[]
    按目前天线排列方式,32TR AAU垂直面最多支持的波束层数为
    \begin{tasks}(4)
        \task 3
        \task 1
        \task 4
        \task 2
    \end{tasks}
\end{choice}


\begin{choice}{A}[]
    关于RRC非活动态的描述,下面说法不正确的是
    \begin{tasks}(2)
        \task 可以5GC寻呼
        \task 可以接收系统消息广播
        \task 可以PLMN选择
        \task 可以小区重选
    \end{tasks}
\end{choice}


\begin{choice}{B}[]
    以下选择中,属于NR空口控制面协议栈的是
    \begin{tasks}(2)
        \task GTP-U
        \task RRC
        \task SCTP
        \task SDAP
    \end{tasks}
\end{choice}


\begin{choice}{C}[]
    5G控制信道资源调度的基本单位CCE和REG之间的正确关系为
    \begin{tasks}(2)
        \task 1CCE = 8REG
        \task 1CCE = 6REG
        \task 1CCE = 7REG
        \task 1CCE = 9REG
    \end{tasks}
\end{choice}

\begin{choice}{B}[]
    损益方程式是经济决策的核心,以下关于损益方程式说法有误的有
    \begin{tasks}(1)
        \task 作业成本法可以使用损益方程式
        \task 作业变动成本法可以使用损益方程式
        \task 损益方程式可表示为:利润=(单价收入-单位产品变动成本)*产品销量-分摊的成本总额
        \task 令损益方程式中利润为0时对应的产品销售量为临界点销售量
    \end{tasks}
\end{choice}

\begin{choice}{B}[]
    NR系统中,小区选择算法里引入了Qoffsettemp参数,该参数是通过哪条消息发送给UE的
    \begin{tasks}(2)
        \task SIB3
        \task SIB2
        \task SIB1
        \task MIB
    \end{tasks}
\end{choice}



\begin{choice}{D}[]
    按照信息传递的方向和时间关系,固定电话通信采用的通信方式为
    \begin{tasks}(4)
        \task 单工通信
        \task 并行传输
        \task 半双工通信
        \task 全双工通信
    \end{tasks}
\end{choice}

\begin{choice}{A}[]
    5G系统进行站内RRC连接重建立时,如果基站拒绝UE重建请求,此时基站会发送哪条信令消息
    \begin{tasks}(4)
        \task RRC establishment
        \task 不回任何消息
        \task RRC Reject
        \task RRC Setup
    \end{tasks}
\end{choice}


\begin{choice}{A}[]
    5G NR中,SSS ID为102,PSS ID为2,此时PCI为
    \begin{tasks}(2)
        \task 302
        \task 308
        \task 1008
        \task 504
    \end{tasks}
\end{choice}



\begin{choice}{A}[]
    NR系统SA组网时,基于切换的EPS Fallback语音互操作B1事件配置目的为
    \begin{tasks}(2)
        \task 异系统数据测量
        \task 异系统切换测量
        \task 异系统语音测量
        \task 系统内切换测量
    \end{tasks}
\end{choice}



\begin{choice}{D}[]
    以下选项中,不属于5G中FR2频段通信优势的是
    \begin{tasks}(2)
        \task 波束集中,提高能量
        \task 可用频段宽
        \task 方向性好,受干扰影响小
        \task 穿透能力强
    \end{tasks}
\end{choice}




\begin{choice}{A}[]
    信息增益在决策树算法中是用来选择特征的指标,生成树时先选择信息增益(\qquad)的特征
    \begin{tasks}(4)
        \task 大
        \task 小
        \task 平均值
        \task 以上都不是
    \end{tasks}
\end{choice}

\begin{choice}{B}[]
    以下选项中,属于NR异系统测量事件的是
    \begin{tasks}(2)
        \task B1
        \task A3
        \task A1
        \task A2
    \end{tasks}
\end{choice}

\begin{choice}{B}[]
    在集成产品开发模式中引入了平台开发的概念,以下不同的产品平台中层级最高的产品平台是
    \begin{tasks}(2)
        \task 产品级平台
        \task 板级平台
        \task  子系统平台
        \task 系统级平台
    \end{tasks}
\end{choice}


\begin{choice}{C}[]
    下面有关PDU会话,流,承载之间关系说法正确的是
    \begin{tasks}(1)
        \task 一个DRB只能映射一个Qos流
        \task 一个UE只能建立一个会话
        \task 一个会话只能建立一个Qos流
        \task 一个Qos流只能映射一个DRB
    \end{tasks}
\end{choice}


\begin{choice}{A}[]
    在C-V2X中,PC5资源分配方式中,终端自主的资源分配模式是指
    \begin{tasks}(2)
        \task mod2
        \task mod4
        \task mod3
        \task mod5
    \end{tasks}
\end{choice}



\begin{choice}{D}[]
    C-V2X中,车与RSU之间的通信接口是
    \begin{tasks}(2)
        \task Xn
        \task PC5
        \task Uu
        \task Ng
    \end{tasks}
\end{choice}


\begin{choice}{A}[]
    5G NR系统中基于基站间的Xn链路切换,基站侧向核心网发的第一条信令为
    \begin{tasks}(1)
        \task HandoverRequest
        \task HandoverRequestAcknowledge
        \task HandoverRequired
        \task SNStatusTransfer
    \end{tasks}
\end{choice}


\begin{choice}{C}[]
    5G通信系统中,下行物理信道PDSCH,PBCH,PDCCH都可以使用的调制方式是
    \begin{tasks}(2)
        \task 64QAM
        \task 256QAM
        \task 16QAM
        \task QPSK
    \end{tasks}
\end{choice}


\begin{choice}{D}[]
    5G无线产品,可用于广域覆盖的是
    \begin{tasks}(2)
        \task PAD
        \task pico
        \task 32TR AAU
        \task DAS
    \end{tasks}
\end{choice}

\section{多选题(共20小题,共80分)}

\begin{choice}{\;ABC\;}[]
    NR系统中,关于5G重叠覆盖带来的主要问题,以下说法正确的是
    \begin{tasks}(2)
        \task 会导致RSRP差
        \task 会频繁切换
        \task 会导致SINR差
        \task 会频繁重选
    \end{tasks}
\end{choice}

\begin{choice}{\;ABCD\;}[]
    5G覆盖测试中,关注的参数有哪些
    \begin{tasks}(2)
        \task CSI-RSRP
        \task SS-RSRP
        \task SS-SINR
        \task CRS-RSRP
    \end{tasks}
\end{choice}

\begin{choice}{\;ABC\;}[]
    关于5G测量,以下说法正确的是
    \begin{tasks}(1)
        \task NR测量中只在异频测量中需要gap。同频测量不需要gap
        \task NR测量中同一个频率即使子载波间隔不同,也需要配置两个不同的MO
        \task 对于NR测量对象中配置的黑列表中的小区,UE不需要进行事件评估和测量
        \task NR测量中测量事件的评估是以beami信号质量进行评估的
    \end{tasks}
\end{choice}

\begin{choice}{\;ABC\;}[]
    以下选项中,属于5G基站功耗增加的原因的选项为
    \begin{tasks}(2)
        \task 大带宽
        \task 通道数增多
        \task 发射功率增加
        \task 多流业务
    \end{tasks}
\end{choice}


\begin{choice}{\;ABCD\;}[]
    关于产品成本存在多种产品核算方法,其中属于管理会计中的产品核算方法为
    \begin{tasks}(2)
        \task 作业变动成本法
        \task 变动成本法
        \task 完全成本法
        \task 作业成本法
    \end{tasks}
\end{choice}

\begin{choice}{\;ACD\;}[]
    以下选项中,属于5G PDSCH信道时域资源分类类型的是
    \begin{tasks}(2)
        \task Type 1
        \task Type 2
        \task Type B
        \task Type A
    \end{tasks}
\end{choice}

\begin{choice}{\;BD\;}[]
    集成产品开发模式在项目管理上的重大变革有哪些
    \begin{tasks}(1)
        \task 成立了IPMT集成产品管理团队和PDT产品开发团队
        \task IPMT专门负责项目的设计/开发流程实施及项目管理
        \task 将项目管理分为IPMT和PDT团队两个层面
        \task PDT专门负责项目管理中的重大决策
    \end{tasks}
\end{choice}


\begin{choice}{\;ABC\;}[]
    以下选项中属于5G大规模天线性能优势的是
    \begin{tasks}(2)
        \task 空间复用增益
        \task 波束赋形增益
        \task 频分复用增益
        \task 分集增益
    \end{tasks}
\end{choice}


\begin{choice}{\;BC\;}[]
    信令就是通信设备之间传递的除用户信息以外的控制信号,信令网就是传输这些控制信号的网络,信令网由哪些部分组成
    \begin{tasks}(2)
        \task 信令链路
        \task 信令点
        \task 信令转接点
        \task 协议信令
    \end{tasks}
\end{choice}


\begin{choice}{\;AD\;}[]
    关于NR PUCCH信道,下列说法正确的是
    \begin{tasks}(1)
        \task PUCCH包含4种格式
        \task PUCCH可以反馈SR
        \task PUCCH可以反馈ACK/NACK
        \task PUCCHformat0可以反馈CSI
    \end{tasks}
\end{choice}


\begin{choice}{\;ABCD\;}[]
    5G智能网联方案中的三网融合解决方案是指哪三网
    \begin{tasks}(2)
        \task 车际网
        \task 传输网
        \task 车内网
        \task 车云网
    \end{tasks}
\end{choice}

\begin{choice}{\;ABCD\;}[]
    NR系统中,PUCCH中UCI携带的信息有
    \begin{tasks}(2)
        \task pointA
        \task 调度请求(Scheduling Request)
        \task ACK/NACK
        \task CSI(Channel State Information)
    \end{tasks}
\end{choice}

\begin{choice}{\;BD\;}[]
    以下哪些属于中国运营商的5G频段
    \begin{tasks}(2)
        \task 1880-1900MHz
        \task 3400-3500MHz
        \task 3500-3600MHz
        \task 4800-4900MHz
    \end{tasks}
\end{choice}

\begin{choice}{\;ABD\;}[]
    NR系统中,RBG是数据信道资源分配的基本调度单位,用于资源分配type0,降低控制信道开销;每个PRB频域RB个数可以是
    \begin{tasks}(4)
        \task 16
        \task 4
        \task 8
        \task 2
    \end{tasks}
\end{choice}

\begin{choice}{\;AD\;}[]
    下面哪些场景会触发系统消息更新
    \begin{tasks}(1)
        \task 收到gNodeB寻呼消息指示有ETWS或CMAS消息广播
        \task 收到gNodeB寻呼消息指示系统消息变化
        \task  UE在开机选择小区驻留
        \task 距离上次正确接收系统消息2小时后
    \end{tasks}
\end{choice}

\begin{choice}{\;BCD\;}[]
    以下选项中,属于5G核心网主要技术特征的是
    \begin{tasks}(2)
        \task 服务化架构
        \task 网络设备虚拟化
        \task 支持网络切片
        \task 支持边缘计算
    \end{tasks}
\end{choice}

\begin{choice}{\;ABD\;}[]
    在5G PDU会话建立过程,PDU Session Resource setup request包含哪些信息
    \begin{tasks}(1)
        \task PDU session Aggregate Maximum Bitrate
        \task Qos ID
        \task PDU-UE-NGAP-ID
        \task 终端业务ip地址
    \end{tasks}
\end{choice}

\begin{choice}{\;ABD\;}[]
    以下选项中,属于MIB信令携带的内容的是
    \begin{tasks}(2)
        \task cellBarred 信息
        \task 公共信道子载波间隔
        \task 系统帧号信息
        \task 调度SIB1的PDCCH信息
    \end{tasks}
\end{choice}

\begin{choice}{\;ABD\;}[]
    针对大唐5G产品,以下选项关于AAU的描述正确的是
    \begin{tasks}(1)
        \task AAU集成了RRU和天线两个模块
        \task AAU具有简化天面,安装方便,加快建网的优点
        \task AAU全称为Active Antenna Unit
        \task AAU架构更有利于天线校准精度,减少由于线缆连接而造成的不可控因素,获得更好的波束赋型性能
    \end{tasks}
\end{choice}

\begin{choice}{\;ABD\;}[]
    关于C-V2X的特点,以下说法正确的是
    \begin{tasks}(2)
        \task 只能短距离通信
        \task 将Uu口和PC5接口相结合
        \task 5G通信低时延,大带宽,海量连接性
        \task 基于成熟的4G,以及复用5G网络,部署成本低
    \end{tasks}
\end{choice}




\section{判断题(共10小题,共30分)}

\begin{choice}{\;正确\;}[]
    GPS跑偏,会对自己或周边站带来干扰
\end{choice}


\begin{choice}{\;错误\;}[]
    RRC连接建立失败会触发RRC重建
\end{choice}

\begin{choice}{\;正确\;}[]
    一个完整的电话网包含了用户终端,交换系统,传输系统等硬件部分,以及实现网络信息控制的信令系统
\end{choice}

\begin{choice}{\;正确\;}[]
    波束扫描的定义为在下行过程中,基站依次使用不同指向的波束发射无线信号
\end{choice}

\begin{choice}{\;正确\;}[]
    TDD中,基站可利用SRS信号评估下行无线环境
\end{choice}

\begin{choice}{\;错误\;}[]
    NR系统中,gNB通过RRC release消息触发UE进入去激活状态,包含TAC消息

\end{choice}


\begin{choice}{\;错误\;}[]
    由于高频段的波长小,高频的衍射和绕射能力都强于低频

\end{choice}

\begin{choice}{\;错误\;}[]
    大唐5G基站设备开通调测时,可以将PC机ip地址配置为172.27.245.100

\end{choice}

\begin{choice}{\;正确\;}[]

    在以解决复杂工程技术问题为目标时,要追求技术指标的最优性
\end{choice}


\begin{choice}{\;错误\;}[]
    采用EPS Fallback机制回落4G时,可采用切换或重定向方式

\end{choice}











