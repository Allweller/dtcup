\chapter{第十届大唐杯B组省赛第一场}
\fancyhead[L]{\color{H6}\kaishu\faIcon{atom}\;第十届「大唐杯」省赛B组真题第一套}
\fancyhead[R]{\color{H6}\kaishu\rightmark\,}

\date{2023年4月08日}{}{\href{https://qm.qq.com/q/UPbGudx8cK}{\textbf{才疏学浅的小熊}}}
{}
{https://ctan.org/pkg/litesolution}{CTAN}
{https://github.com/xubenshan}{GitHub}

注:答案仅供参考,若有问题,请联系我。QQ号:1782622988

\section{单选题(共30道题,共90分)}



\begin{choice}{B}[节能]
    5G控制信道资源调度的基本单位CCE和REG之间的正确关系为
    \begin{tasks}(1)
        \task 1CCE = 7RBG
        \task 1CCE = 9RB
        \task 1CCE = 6REG
        \task 1CCE = 8RE
    \end{tasks}
\end{choice}


\begin{choice}{B}[5G无线技术]
    在进行5G覆盖率优化过程中,对一些MR弱覆盖小区的覆盖优化措施描述不准确的是
    \begin{tasks}(1)
        \task 优化天线权值,权值和覆盖场景准确匹配
        \task 功率优化,提升覆盖的深度
        \task 减小天线覆盖的下倾角,增加覆盖范围
        \task 天馈调整,天线主瓣朝向用户密集的区域
    \end{tasks}
\end{choice}




\begin{choice}{A}[5G无线技术]
    5G协议栈中,下列选项中不属于是NR层二子层的是
    \begin{tasks}(4)
        \task RLC
        \task PDCP
        \task SDAP
        \task NAS
    \end{tasks}
\end{choice}


\begin{choice}{D}[5G协议与信令]
    下面降低元器件材料清单的变动成本方法不正确的是
    \begin{tasks}(2)
        \task 技术模块的技术重用
        \task 用硬件代替软件
        \task 利用摩尔定律降低系统成本
        \task 元器件的归一化
    \end{tasks}
\end{choice}

% \begin{solution}
%     R15版本是第一代5G的版本。PDSCH(物理下行数据信道)支持的最大调制方式为256QAM。在R16版本中又增加了1024QAM调制。

%     补充:下行不同信道的调制方式:
%     \begin{itemize}
%         \item PBCH(物理广播信道)QPSK
%         \item PDCCH(物理下行控制信道)QPSK
%         \item PDSCH(物理下行数据信道)QPSK,16QAM,64QAM,256QAM,1024QAM

%               \sokka{D}
%     \end{itemize}
% \end{solution}
\begin{choice}{B}[节能]
    5G NR PRACH格式(format3)适合于如下哪种场景
    \begin{tasks}(2)
        \task 深度覆盖
        \task 高速场景
        \task 沙漠等环境
        \task 市区,农村和郊区等环境
    \end{tasks}
\end{choice}

\begin{choice}{D}[]
    符号关断是主要的节能技术,符号关断节能的本质是
    \begin{tasks}(2)
        \task 降低PA静态功耗
        \task 以上三种说法均错误
        \task 降低PA动态功耗
        \task 降低BBU功耗
    \end{tasks}
\end{choice}




\begin{choice}{D}[最大分辨力]
    在5G系统中:组网频段为3.5GHz,系统带宽为100M,帧结构为2.5ms双周期,则时域频域可调度资源分别为
    \begin{tasks}(2)
        \task 时域DL:1400/UL:600 频域:273RB
        \task 时域DL:1000/UL:1000 频域:5M:28RB 10M:52RB
        \task 时域DL:1000/UL:1000 频域:20M:106RB 40M:216RB
        \task 时域DL:1000/UL:10000 频域:20M:106RB 30M:160RB
    \end{tasks}
\end{choice}


\begin{choice}{B}[双缝干涉]
    5G协议栈中,负责QoS流到DRB映射的是
    \begin{tasks}(4)
        \task RRC
        \task SDAP
        \task PDCP
        \task RLC
    \end{tasks}
\end{choice}
%    \switchcolumn\centering
%    \vfill
%    \begin{tikzpicture}[decoration={markings,mark=between positions .2 and .8 step 18mm with {\arrow{stealth}}}]
%        \draw [very thick] (0,1.5)--(0,-1.5);
%        \draw [thick,postaction=decorate] (-1,0)--(0,0.5)--(2,0.8);
%        \draw [thick] (0,0.5)--(0.8,0);
%        \draw [thick,postaction=decorate] (-1,0)--(0,-0.5)--(2,0.8);
%        \draw [very thick] (0,0)--(1.8,0);
%        \draw [very thick] (2,1.8)--(2,-1.8);
%        \fill [pattern=north west lines] (0,0) rectangle (1.8,-0.2);
%        \fill [pattern=north east lines] (2,1.8) rectangle (2.2,-1.8);
%        \node [anchor=east] at (-1,0) {$S$} node [anchor=south east] at (0,0.5) {$S_1$} node [anchor=north east] at (0,-0.5) {$S_2$};
%        \node [anchor=north] at (0.8,-0.2) {$M$} node [anchor=west] at (2.2,0.8) {$P$} node [anchor=north] at (2.1,-1.8) {$E$};
%        \node at (-1,0) {$\times$};
%    \end{tikzpicture}
%    \vfill
%\end{paracol}
%\vspace{-.75em}


\begin{choice}{A}[迈克尔逊干涉仪]
    2.6G组网时,SSB波束配置最大为
    \begin{tasks}(4)
        \task 7
        \task 4
        \task 2
        \task 8
    \end{tasks}
\end{choice}


\begin{choice}{A}[光栅]
    5G NR系统中。100M带宽组网,30子载波间隔,单小区最大支持的PRB为
    \begin{tasks}(1)
        \task 273PRB
        \task 175PRB
        \task 100PRB
        \task 200PRB
    \end{tasks}
\end{choice}

\begin{choice}{A}[]
    5G NR时频资源,RBG(Resource Block Group)频域不可配置为
    \begin{tasks}(4)
        \task 3个RB
        \task 2个RB
        \task 4个RB
        \task 8个RB
    \end{tasks}
\end{choice}


\begin{choice}{A}[]
    在5G系统中,700M组网时,目前实现的最大流数是
    \begin{tasks}(2)
        \task UL 2流 DL 4流
        \task UL 1流 DL 2流
        \task UL 2流 DL 2流
        \task UL 4流 DL 2流
    \end{tasks}
\end{choice}


\begin{choice}{B}[]
    网络使用x86架构的通用设备替代专用设备的网络技术为
    \begin{tasks}(4)
        \task MEC
        \task SDN
        \task NFV
        \task 切片
    \end{tasks}
\end{choice}


\begin{choice}{C}[]
    在5G覆盖规划中,对于高容量,大面积高价值场景优先选择如下哪种方案
    \begin{tasks}(2)
        \task Pinsite
        \task DAS
        \task Slsite
        \task 宏覆盖
    \end{tasks}
\end{choice}

\begin{choice}{B}[]
    在5G系统中,在手机开机流程中,负责业务承载建立的过程是
    \begin{tasks}(1)
        \task 安全模式建立过程
        \task PDU会话建立过程
        \task 初始上下文建立过程
        \task RRC连接建立过程
    \end{tasks}
\end{choice}

\begin{choice}{B}[]
    5G NR系统中,主系统信息块MIB的传输时间间隔为
    \begin{tasks}(4)
        \task 80ms
        \task 40ms
        \task 160ms
        \task 10ms
    \end{tasks}
\end{choice}



\begin{choice}{D}[]
    C-V2X的标准化可分为多个阶段进行,其中2017年正式发布的支持LTE-V2X的3GPP协议版本是
    \begin{tasks}(4)
        \task R13
        \task R14
        \task R16
        \task R15
    \end{tasks}
\end{choice}

\begin{choice}{A}[]
    NR系统中,调度SIB1的搜索空间类型为
    \begin{tasks}(2)
        \task Type0-PDCCH
        \task Type0A-PDCCH
        \task Type2-PDCCH
        \task Type1-PDCCH
    \end{tasks}
\end{choice}


\begin{choice}{A}[]
    符号关断是主要的节能技术,符号关断节能的本质是
    \begin{tasks}(2)
        \task 降低PA静态功耗
        \task 以上三种说法均错误
        \task 降低BBU功耗
        \task 降低PA动态功耗
    \end{tasks}
\end{choice}



\begin{choice}{A}[]
    NR中定义了搜索空间的概念,以下选中用于解调随机接入过程中MSG4的搜索空间为
    \begin{tasks}(2)
        \task Type0A-PDCCH
        \task Type0-PDCCH
        \task Type1-PDCCH
        \task Type2-PDCCH
    \end{tasks}
\end{choice}



\begin{choice}{D}[]
    在产品设计开发流程中如果发现自己的核心技术已被他人申请了专利,这是要启动专利应对机制进行专利决策,交叉许可属于应对机制的一种,关于交叉许可说法正确的为
    \begin{tasks}(1)
        \task  交叉许可是通过修改原来的设计/开发解决方案,从而避免已被他人申请的专利
        \task 交叉许可专利应对机制是专利避让和交纳专利使用费的叠加效果
        \task 交叉许可是通过在同一类产品中申请竞争对手难以被避让,绕开的发明专利,形成在同一类产品中彼此侵权的局面,相互依赖,进行低成本的交叉许可
        \task 交叉许可是通过与专利权人签署专利授权使用协议,通过向专利权人交纳专利使用费的方式来化解专利侵权的纠纷
    \end{tasks}
\end{choice}




\begin{choice}{A}[]
    5G空口协议栈中,物理层和MAC层之间的服务接口是
    \begin{tasks}(4)
        \task 传输信道
        \task 无线承载
        \task 逻辑信道
        \task RLC信道
    \end{tasks}
\end{choice}

\begin{choice}{B}[]
    5G系统中,从下列哪条信令中可知终端是否支持5G双连接
    \begin{tasks}(2)
        \task RRC Connection Setup complete
        \task RRC:UE Capability Enquiry
        \task UE Capability Information
        \task RRC Connection Reconfiguration
    \end{tasks}
\end{choice}

\begin{choice}{B}[]
    在5G NR网络中PBCH的DMRS的频域位置和以下选项中哪个参数相关
    \begin{tasks}(2)
        \task SI-RNTI
        \task Cell ID
        \task  Bandwidth
        \task PCI
    \end{tasks}
\end{choice}


\begin{choice}{C}[]
    优化成本是经济决策的核心,对生产作业环节的成本起决定性作用的因素为
    \begin{tasks}(2)
        \task 产品价格
        \task 经济回报
        \task 市场竞争
        \task 设计解决方案
    \end{tasks}
\end{choice}


\begin{choice}{A}[]
    5G核心网中,NAS(N1)信令(MM消息)的终结点为
    \begin{tasks}(4)
        \task UDR
        \task UPF
        \task AMF
        \task SMF
    \end{tasks}
\end{choice}



\begin{choice}{D}[]
    以下关于CORESET的描述错误的是
    \begin{tasks}(2)
        \task CORESET的频率分配必须是连续的
        \task 每个BWP最多可以配置3个CORESET
        \task CORESET频域上以6个RBs为粒度进行配置
        \task CORESET配置跨越1-3个连续的OFDM符号
    \end{tasks}
\end{choice}


\begin{choice}{A}[]
    在LTE-V2X中,基于PC5接口,基站集中调度分配模式是
    \begin{tasks}(4)
        \task mod2
        \task mod4
        \task mod3
        \task mod5
    \end{tasks}
\end{choice}


\begin{choice}{C}[]
    关于5G的RRC状态转换说法不正确的选项是
    \begin{tasks}(1)
        \task RRC空闲态可以到RRC非活动态
        \task RRC连接可以到RRC非活动态
        \task RRC连接可以到RRC空闲态
        \task RRC空闲可以到RRC连接
    \end{tasks}
\end{choice}


\begin{choice}{D}[]
    C-V2X系统中,BSM消息指的是
    \begin{tasks}(2)
        \task 信号灯消息
        \task 路测分发交通参与者实时消息
        \task 车辆基本安全类消息
        \task 地图消息
    \end{tasks}
\end{choice}

\section{多选题(共20小题,共80分)}

\begin{choice}{\;ABC\;}[]
    CORESET:control-resource0set,一组物理资源集合,时域上的OFOM符号数可以是
    \begin{tasks}(4)
        \task 1
        \task 3
        \task 4
        \task 2
    \end{tasks}
\end{choice}

\begin{choice}{\;ABCD\;}[]
    根据C-V2X标准定义,以下哪些属于车联网工作模式
    \begin{tasks}(4)
        \task V2P
        \task V2N
        \task V2V
        \task V2I
    \end{tasks}
\end{choice}

\begin{choice}{\;ABC\;}[]
    以下选项中属于为解决复杂技术问题的方式特点为
    \begin{tasks}(1)
        \task 不考虑用户需求
        \task 不惜牺牲系统的可靠性
        \task 不考虑非技术因素的边界条件
        \task 不惜成本代价
    \end{tasks}
\end{choice}

\begin{choice}{\;ABC\;}[]
    以下选项中,属于5G下行物理信道的选项是
    \begin{tasks}(2)
        \task PDSCH
        \task PDCCH
        \task PCH
        \task PBCH
    \end{tasks}
\end{choice}


\begin{choice}{\;ABCD\;}[]
    5G NR系统中,UE在RRC INACTVE状态时被配置了一个RNA,以下关于RNA描述正确的是
    \begin{tasks}(2)
        \task RAN与TAU概念相同
        \task RANU过程就是TAU过程
        \task UE移出了配置RNA会触发RANU
        \task RNA Update timer超时会触发RANU
    \end{tasks}
\end{choice}

\begin{choice}{\;ACD\;}[]
    在5G基站日常维护过程中,导致AC校准问题的原因可能是
    \begin{tasks}(2)
        \task 输出功率问题
        \task 驻波比问题
        \task 光链路异常问题
        \task GPS时钟源异常问题
    \end{tasks}
\end{choice}

\begin{choice}{\;BD\;}[]
    以下选项中哪些属于5G测量报告中携带的内容
    \begin{tasks}(2)
        \task measID
        \task ServingCell PCI
        \task reportType
        \task NeighCell PCI
    \end{tasks}
\end{choice}


\begin{choice}{\;ABC\;}[]
    下面有关网络切片说法正确的是
    \begin{tasks}(1)
        \task 在一个硬件基础设施切分出多个虚拟的端到端网络
        \task 在一个硬件基础设施切分出多个物理的端到端网络
        \task 每个切片之间是物理上隔离
        \task 每个切片之间是逻辑上隔离
    \end{tasks}
\end{choice}


\begin{choice}{\;BC\;}[]
    UE在下面哪些场景会主动读取系统消息
    \begin{tasks}(2)
        \task 重选小区
        \task 开机选择小区驻留
        \task 从非覆盖区返回到覆盖区时
        \task 切换完成
    \end{tasks}
\end{choice}


\begin{choice}{\;AD\;}[]
    5G NR中,毫米波支持的信道带宽是
    \begin{tasks}(4)
        \task 200Mhz
        \task 10Mhz
        \task 50Mhz
        \task 100Mhz
    \end{tasks}
\end{choice}


\begin{choice}{\;ABCD\;}[]
    以下选项中属于MIB信令携带的内容的是
    \begin{tasks}(2)
        \task 系统帧号信息
        \task cellBarred信息
        \task 调度SIB1的PDCCH信息
        \task 公共信道子载波间隔
    \end{tasks}
\end{choice}

\begin{choice}{\;ABCD\;}[]
    5G系统日常维护,现场系统人员主要工作内容包括
    \begin{tasks}(2)
        \task 设备巡检
        \task 专项核查
        \task 指标监控
        \task 告警分析
    \end{tasks}
\end{choice}

\begin{choice}{\;BD\;}[]
    5G NR系统中,适用于N41频段的信道栅格为
    \begin{tasks}(4)
        \task 60kHz
        \task 15kHz
        \task 100kHz
        \task 30kHz
    \end{tasks}
\end{choice}

\begin{choice}{\;ABD\;}[]
    以下选项中,属于5G核心网的主要特征的是
    \begin{tasks}(2)
        \task 支持边缘计算
        \task 支持网络切片
        \task 服务化架构
        \task 网络设备虚拟化
    \end{tasks}
\end{choice}

\begin{choice}{\;AD\;}[]
    大唐5G基站设备维护时,需要使用的工具包括
    \begin{tasks}(1)
        \task 斜口钳
        \task 十字螺丝刀
        \task  光功率计
        \task 万用表
    \end{tasks}
\end{choice}

\begin{choice}{\;BCD\;}[]
    5G NR中,SSB由以下哪几个部分组成
    \begin{tasks}(4)
        \task PBCH
        \task PSS
        \task SSS
        \task CSI-RS
    \end{tasks}
\end{choice}

\begin{choice}{\;ABD\;}[]
    以下选项中属于5G中PDU会话属性的参数有
    \begin{tasks}(2)
        \task S-NSSAI
        \task PDU会话类型
        \task PDU会话ID
        \task 数据网络名称
    \end{tasks}
\end{choice}

\begin{choice}{\;ABD\;}[]
    作业成本法将成本分配到成本对象有下面哪几种形式
    \begin{tasks}(2)
        \task 未来预估
        \task 成本追溯
        \task 动因归集
        \task 分摊
    \end{tasks}
\end{choice}

\begin{choice}{\;ABD\;}[]
    以下哪些措施可以达到降低元器件的采购成本的目的
    \begin{tasks}(1)
        \task 元器件的归一化
        \task 降低模块,部件的共用度
        \task 超大规模集成电路的选用
        \task 尽量采用独家垄断的元器件
    \end{tasks}
\end{choice}

\begin{choice}{\;ABD\;}[]
    传统产品开发模式在项目管理机制上的弊端有哪些
    \begin{tasks}(2)
        \task 各部门的职责,考核内容各不相同
        \task 各业务执行单位只对本部分负责,而不是对产品,对客户负责
        \task 铁路警察,各管一段
        \task 各业务执行部门的职责严格区分
    \end{tasks}
\end{choice}




\section{判断题(共10小题,共30分)}

\begin{choice}{\;正确\;}[]
    多步式利润表中销售额发生的标志是合同中产品与服务的所有权已发生转移
\end{choice}


\begin{choice}{\;错误\;}[]
    共建共享采用独立载波共享,此时双方运营商是各自独立小区
\end{choice}

\begin{choice}{\;正确\;}[]
    无线网的切片间可以CU-C共享,CU-U隔离
\end{choice}

\begin{choice}{\;正确\;}[]
    NR系统中,常规CP配置下,一个时隙一共12个符号
\end{choice}

\begin{choice}{\;正确\;}[]
    5G NR系统中,初始随机接入失败后,终端可以提升发射功率,继续发起随机接入过程
\end{choice}

\begin{choice}{\;错误\;}[]
    5G所有场景都支持自包含帧结构

\end{choice}


\begin{choice}{\;错误\;}[]
    5G无线系统中,NSA Option3是5G网络成熟阶段的目标架构

\end{choice}

\begin{choice}{\;错误\;}[]
    中移如果使用2.6GHz部署5G室分,那么在原有的DAS系统直接合路即可

\end{choice}

\begin{choice}{\;正确\;}[]

    5G系统中,多条Qos Flow可以映射到同一个DRB
\end{choice}


\begin{choice}{\;错误\;}[]
    大唐5G设备EMB6216的高度为3U

\end{choice}











